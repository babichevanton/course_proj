% Класс документа с параметрами: кегль шрифта, шрифт с засечками для формул,
% соотношение сторон слайда
\documentclass[14pt,mathserif,aspectratio=43]{beamer}

% Кодировка файла
\usepackage[utf8]{inputenc}
% Переносы для русского языка
\usepackage[russian]{babel}

\usepackage{xcolor}
% Ищется в яндексе по запросу «травяной цвет»
\definecolor{grass}{RGB}{93,161,48}

\usepackage[T1]{fontenc}
% Шрифт для формул
\usepackage[bitstream-charter]{mathdesign}

% Математические символы
\usepackage{amsmath}

% Перечёркивание текста
\usepackage{ulem}

\usepackage{bbm}

\usepackage{epstopdf}

% Основной цвета beamer'а (по умолчанию — синий)
\setbeamercolor{structure}{fg=black}

% Путь к картинкам
\graphicspath{{images/}}

% Нумерация слайдов
\setbeamertemplate{footline}{%
\raisebox{7pt}{\makebox[\paperwidth]{%
\hfill\makebox[10pt]{\scriptsize\textcolor{gray}{\insertframenumber~~~~}}}}}

% Начинать нумерацию слайдов не с титульного слайда
% \let\otp\titlepage
% \renewcommand{\titlepage}{\otp\addtocounter{framenumber}{-1}}

% Сноска без номера
\newcommand\articlenote[1]{%
  \begingroup%
  \renewcommand\thefootnote{}\footnote{#1}%
  \addtocounter{footnote}{-1}%
  \endgroup%
}

% Убрать навигацию beamer'а по умолчанию
\beamertemplatenavigationsymbolsempty

% Параметры для создания титульной страницы
\title{Извлечение информации из неструктурированных источников с использованием имеющихся табличных данных \vspace{-1em}}
\author{Антон Бабичев}

\institute{

    Научный руководитель Недумов Ярослав Ростиславович\\
    \medskip
    Факультет ВМК МГУ им. М.В.\,Ломоносова\\
    Кафедра Системного Программирования\\

    \medskip
}

% Время
\date{\small{20 мая 2014 г.}}

\begin{document}

%-------------------------------------------------------------------------------
\begin{frame}[plain]
    \titlepage
\end{frame}

\begin{frame}{Неформальная постановка задачи}
	Дан неструктурированный текстовый пост --- описание товара в интернет-магазине.\\
	\medskip
	Например, ``HP ENVY TouchSmart 15.6 inches Touch-Screen Laptop Intel Core i5 8GB Memory 750GB Hard Drive Natural Silver''\\
	\medskip
	Необходимо исследовать его содержимое, руководствуясь структурированной базой знаний о товарах того же типа.\\
\end{frame}

\begin{frame}{Применение}
    \begin{itemize}
	    \item Построение базы знаний по содержанию веб-страниц 

	    \item Связывание данных на разных страницах, посвященных одному и тому же объекту

		\item Извлечение дополнительных сведений об объекте(цена товара, дата публикации и т.д.) 
    \end{itemize}
\end{frame}

\begin{frame}{Формальная постановка задачи}
	Входные данные: текстовый пост как набор токенов $\{t_1, t_2, \ldots t_n\}$\\
	База знаний: множество наборов пар $\{attr: val\}$\\
    Один набор отражает информацию об одной единице товара\\
	Выходные данные: набор меток-атрибутов для токенов поста либо метка ``пустышка'' $\{tok: attr | junk\}$\\
\end{frame}

\begin{frame}{Задачи}
    \begin{itemize}
	    \item Провести обзор существующих методов извлечения структурированных данных

	    \item Выбрать наиболее перспективный метод и реализовать его

		\item Протестировать метод на домене, отличном от авторского 
    \end{itemize}
\end{frame}

\begin{frame}{Методы решения}
    \begin{itemize}
	    \item Методы на основе NLP\\
	    Предполагается, что текст написан на естественном языке

	    \item Методы на основе Information Extraction\\
	    Предполагается наличие какой-либо структуры текста

		\item Метод на основе множества элементарных исходов\\
		Разработан специально для данной задачи

		\articlenote{Michelson, Knoblock, Creating Relational Data from Unstructured and Ungrammatical Data Sources, 2008} 
    \end{itemize}
\end{frame}

\begin{frame}{Использованные данные}
    \begin{itemize}
    	\item Домен: описания ноутбуков на английском языке

	    \item База знаний: 8055 записей с \textit{www.amazon.com}\\
	    Запись --- описание и набор характеристик

	    \item Данные для тестирования: 104 размеченных вручную записей с \textit{www.bestbuy.com}\\ 
	    Запись --- описание с набором меток для токенов
    \end{itemize}
\end{frame}

\begin{frame}{Описание метода}
    \begin{enumerate}
    	\item Предобработка

	    \item Связывание

	    \item Разметка
    \end{enumerate}

\end{frame}

\begin{frame}{Описание метода --- предобработка}
    	Приблизительное сравнение поста с элементами базы знаний $\to$ подмножество базы знаний\\
    	Записи базы знаний кластеризуются по группам атрибутов.\\
    	Для описания одного кластера используется правило в виде ДНФ:\\
    	\[
    	(\{attr_1 = val_1\} \land \{attr_2 = val_2\}) \vee \{attr_3 = val_3\} \vee \ldots
    	\]
    	\medskip
    	Для оптимального разбиения используется алгоритм BlockSchemeLearner.\\
\end{frame}

\begin{frame}{Описание метода --- Связывание}
    	Подмножество базы знанй $\to$ единственный элемент базы знанй --- схема поста\\
    	$\{post,candidate\} \to$  дескриптор\\
		Дескриптор $\to$ SVM классификатор\\
\end{frame}

\begin{frame}{Описание метода --- Разметка}
    	Токены поста $\to$ массив меток атрибутов и ``пустышки''\\
		$\{$токен, схема поста$\}$ $\to$ дескриптор $\to$ Multi-Class SVM классификатор\\
		Multi-Class SVM классификатор обучен ставить токену в соответствие метку атрибута\\
		После проводится процедура чистки размеченных токенов, где отсеиваются ``пустышки''\\
\end{frame}

\begin{frame}{Реализация}
		В качестве языка реализациии был выбран язык программирования \textit{Java}\\
		Для классификации используется библиотека \textit{libsvm}\\
		\medskip
		Разработаны и реализованы 2 основных класса:\\
	    \begin{itemize}
	    	\item BloskSchemeLearner выполняет предобработку базы знаний

		    \item PostExplorer проводит связывание и разметку поста
	    \end{itemize}
    	Загрузка данных осуществлена с помощью краулера, написанного на Python с использованием библиотеки scrapy\\
\end{frame}

\begin{frame}{Результаты}
	\begin{itemize}
		\item Был реализован и протестирован метод, описанный в статье

		\item Тестирование проводилось для шести атрибутов ноутбуков (производитель, диагональ, цвет\ldots)

		\item В среднем $F_1= 0.67, precision = 0.58, recall = 0.82$

		\item Лучший результат (диагональ): $F_1= 0.94, precision = 0.92, recall = 0.96$

		\item Худший результат (линейка): $F_1= 0.38, precision = 0.26, recall = 0.66$

		\item Основные ошибки связаны со спецификой выборки и процедурой чистки
	\end{itemize}
\end{frame}

\begin{frame}{Дальнейшие планы}
	\begin{itemize}
		\item Адаптация метода --- изменение дескриптора, классификаторов, поддержка русского языка и т.д.
		\item Улучшение процедуры чистки размеченных токенов
		\item Тестирование на большей базе
	\end{itemize}
\end{frame}

\begin{frame}{}
	\begin{center}
		Спасибо за внимание!\\
	\end{center}
\end{frame}

\end{document}

