\chapter{Обзор существующих решений}


\section{Методы, основанные на распозавании именованных сущностей}

Это группа методов, в основе которых лежат техники распознавания именованных сущностей. Именованная сущность - это группа слов в тексте, которая описывает реальный объект. Например, ``Apple Inc.'', ``John Broun'', ``information extraction'' и т.д. Поиск каких-либо именованных сущностей ведется в тексте с помощью паттерна. По способу нахождения паттерна методы делятся на подходы, основанные на правилах, и на статистические подходы.

Методы, основанные на правилах, (например, [1]) находят паттерн в тексте, редуцируя обобщенные правила. Например, из правила ``является числом'' получаются более специфичные правила ``является 4-значным числом'' или ``является дробным числом''. Для вычисления правил используются размеченная вручную обучающая база, поэтому для больших текстов данный подход неприменим.

Статистические подходы (например, система Nymble [2]) используют вариации EM-алгоритма для нахождения распределений токенов по сущностям. В частности, в Nymble использубтся скрытые Марковские модели. Поскольку предположение, что токены сущности распределены по нормальному закону во всем тексте может быть неприемлемо в случае неструктурированного источника, данный подход нам также неинтересен.

\section{Методы, основанные на NLP}

Методы (например, [3]) этой группы используют в вычислениях предположение о наличии структуры естественного языка в текстовом источнике, поэтому для нашей задачи не годятся.


\section{Методы, основанные на использовании базы знаний}

Данная группа методов представлена подходом, описанным Matthew Michelson и Craig A. Knoblock [4] и характеризуется использованием при анализе содержания текста некой базы знаний об объектах какого-либо типа. Анализируемый текст авторы подхода рассматривают просто как набор токенов без какой-либо структуры --- так называемый ``текстовый пост''. База знаний представлена записями об объектах в виде совокупности пар \{атрибут: значение\} и в рамках статьи называется ``множество элементарных исходов''. В своей статье авторы описывают рещение задачи разметки токенов поста заданным набором меток-атрибутов и особой метки ``junc'', символизирующей непринадлежность токена ни одному из атрибутов объекта.

Решается задача в несколько этапов: 
\begin{enumerate}
	\item Из множества элементарных исходов с помощью приблизительного сравнения выбирается подмножество записей.
	\item По посту и каждой отобранной записи строится вектор признаков
	\item Вектор признаков попадет на вход SVM-классификатору, обученному относить запись к одному из 2 классов: \textit{запись является схемой поста} и \textit{запись не является схемой поста}
	\item Лучший кандидат становится схемой поста
	\item По схеме поста и каждому токену поста строится вектор признаков
	\item Вектор признаков попадет на вход Multi-Class SVM-классификатору, обученному каждому токену поста ставить в соответствие метку-атрибут
	\item Для группы токенов одного атрибута проводится чистка --- отсеиваются токены с меткой ``junc''
	\item Остается размеченный пост - требуемый результат работы.
\end{enumerate}

Результат работы алгоритма меряется $F_1 \text{мерой}$ и составляет 0.7704.

Данный подход интересен тем, что решает задачу извлечения информации для полностью лишенного структуры источника. Также для обучения классификаторов не требуется размеченных вручную данных.