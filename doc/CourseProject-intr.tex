\chapter*{Введение}
\addcontentsline{toc}{chapter}{Введение}

Анализ информации является важной задачей во многих отраслях человеческой деятельности, таких как наука, производство, экономика. Информации в наши дни очень много, и накапливается она усиленными темпами. Однако большинство информации имеет неудобную для интерпретации форму. В частности, текст может быть абсолютно неструктурирован и не подчинен каким-либо правилам. Работа с такими данными занимает большое количество времени, поэтому естественно желание упростить, ускорить и автоматизировать эту задачу. Далее эти данные используются для дальнейшего анализа.

Наибольшее количество информации сосредоточено в Интернете, и в основном она представлена в виде web-страниц. Данные об одном и том же объекте могут встречаться на разных сайтах, и информация, которую в себе несут эти данные, может друг друга дополнять. Использование единой структурированной базы знаний об объектах какого-либо типа очень полезно и упрощает анализ неструктурированных данных о них. Информация, полученная в результате анализа такого неструктурированного текста, несет пользу и помогает дополнить наши знания об определенном объекте. Например, мы можем дополнить базу знаний записью о новом объекте, если его параметры схожи по типу с таковыми у объектов в базе знаний. Либо мы можем дополнить запись об уже существующем в базе объекте новыми параметрами-характеристиками объекта. Также полезно уметь свзывать данные об одном и том же объекте с разных страниц в Интернете, чтобы, например, проверить имеющиеся знания в базе либо же дополнить ее.

В данной курсовой работе сконцентрировано внимание на извлечении информации из неструктурированного источника с использованием базы знаний